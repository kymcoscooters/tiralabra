\documentclass[a4paper]{article}
\usepackage[utf8]{inputenc}
\usepackage[T1]{fontenc}
\usepackage[finnish]{babel}
\usepackage{geometry}
\usepackage{amsmath}
\usepackage{amsthm}
\usepackage{amssymb}
\begin{document}

\part*{Määrittelydokumentti}

Tavoitteena olisi luoda ohjelma joka tiivistää tekstitiedostoja ja purkaa jo tiivistettyjä tiedostoja luettavaan  muotoon. Käytetty algoritmi tässä on Huffman, ja toteutukseen kuuluu tietorakenteena binääripuu. Toteutan ohjelman Javalla.
\\
\\
Ohjelma saa siis syötteenä jonkun teksti-tiedoston, tiivistää sen ja palauttaa/tallentaa siitä tiivistetyn version. Ohjlema voisi myös jotenkin ilmaista kuinka paljon tiedostoa on saatu kompressoitua.
\\
\\
Toteutan mahdollisesti ohjelman niin että siitä tulee verkkosivu, johon voi lataa tekstitiedoston, ja prosessoinnin jälkeen ladata koneelle tiivistetyn version, ja toisin päin.
\\
\\
Tavoitteena olisi saada aikavaativuus $\mathcal{O}(n\log{}n)$. Tilavaativuuden suhteen tavoite on että tiivistetty tiedosto olisi noin puolet pienempi kuin alkuperäinen.
\end{document}