\documentclass[a4paper]{article}
\usepackage[utf8]{inputenc}
\usepackage[T1]{fontenc}
\usepackage[finnish]{babel}
\usepackage{geometry}
\usepackage{amsmath}
\usepackage{amsthm}
\usepackage{amssymb}
\begin{document}

\part*{Viikkoraportti 1}

Erinäisistä syistä en päässyt aloittamaan projektin tekoa ensimmäisellä viikolla, joten tämän ensimmäisen viikon viikkoraportti on oikeasti raportti toisen viikon alusta. Tein alkuviikosta ne asiat jotka kuuluivat ensimmäisen viikon palautukseen.
\\
\\
Olen siis valinnut aiheen, eli ohjelma joka tiivistää tekstitiedostoa käyttäen Huffmanin algoritmia. Olen luonut github-repon ja rekisteröinyt itseni Labtooliin.
\\
\\
En ole vielä aloittanut ohjelman toteuttamista, joten ohjelma ei ole vielä edistynyt ollenkaan.
\\
\\
Vaikeuksia ei ole ollut, aloitin tämän kurssin keväällä 2018, mutta joudun hyppäämään siitä alussa pois, ja ajattelin nyt käyttää samaa aihetta mitä silloinkin oli tarkoitus.
\\
\\
Seuraavaksi tulee sovelluksen rungon teko, netbeans-projektin luonti yms.
\end{document}